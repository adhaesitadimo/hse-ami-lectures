% Размер страницы и шрифта
\documentclass[12pt,a4paper]{article}

%% Работа с русским языком
\usepackage{cmap}                   % поиск в PDF
\usepackage{mathtext}               % русские буквы в формулах
\usepackage[T2A]{fontenc}           % кодировка
\usepackage[utf8]{inputenc}         % кодировка исходного текста
\usepackage[english,russian]{babel} % локализация и переносы

%% Изменяем размер полей
\usepackage[top=0.5in, bottom=0.75in, left=0.625in, right=0.625in]{geometry}

%% Различные пакеты для работы с математикой
\usepackage{mathtools}  % Тот же amsmath, только с некоторыми поправками
\usepackage{amssymb}    % Математические символы
\usepackage{amsthm}     % Пакет для написания теорем
\usepackage{amstext}
\usepackage{array}
\usepackage{amsfonts}
\usepackage{icomma}     % "Умная" запятая: $0,2$ --- число, $0, 2$ --- перечисление

%% Графика
\usepackage[pdftex]{graphicx}
\graphicspath{{images/}}

%% Прочие пакеты
\usepackage{listings}               % Пакет для написания кода на каком-то языке программирования
\usepackage{algorithm}              % Пакет для написания алгоритмов
\usepackage[noend]{algpseudocode}   % Подключает псевдокод, отключает end if и иже с ними
\usepackage{indentfirst}            % Начало текста с красной строки
\usepackage[colorlinks=true, urlcolor=blue]{hyperref}   % Ссылки
\usepackage{pgfplots}               % Графики
\pgfplotsset{compat=1.12}
\usepackage{forest}                 % Деревья
\usepackage{titlesec}               % Изменение формата заголовков
\usepackage[normalem]{ulem}         % Для зачёркиваний
\usepackage[autocite=footnote]{biblatex}    % Кавычки для цитат и прочее
\usepackage[makeroom]{cancel}       % И снова зачёркивание (на этот раз косое)

% Изменим формат \section и \subsection:
\titleformat{\section}
	{\vspace{1cm}\centering\LARGE\bfseries} % Стиль заголовка
	{}                                      % префикс
	{0pt}                                   % Расстояние между префиксом и заголовком
	{}                                      % Как отображается префикс
\titleformat{\subsection}                   	% Аналогично для \subsection
	{\Large\bfseries}
	{}
	{0pt}
	{}

% Поправленный вид lstlisting
\lstset { %
    backgroundcolor=\color{black!5}, % set backgroundcolor
    basicstyle=\footnotesize,% basic font setting
}

% Теоремы и утверждения. В комменте указываем номер лекции, в которой это используется.
\newtheorem*{hanoi_recurrent}{Свойство} % Лекция 1
\let\epsilent\varepsilon                % Лекция 8
\DeclareMathOperator{\rk}{rank}         % Лекция 20

% Информация об авторах
\author{Группа лектория ФКН ПМИ 2015-2016 \\
	Никита Попов \\
	Тамерлан Таболов \\
	Лёша Хачиянц}
\title{Лекции по предмету \\
	\textbf{Алгоритмы и структуры данных}}
\date{2016 год}


\begin{document}

\section*{Лекция ?? от 15.03.2016}

\subsection{?}

Пусть у нас есть некоторая функция Init$(n)$, создающая массив из $n$ нулей;
Read$(i)$;
Write$(a, i)$

Также будем исходить из предположения, что памядь под $n$ элементов можно выделить за время $O(1)$; чтение и запись тоже будем считать константными.

Соответственно Init$(n)$ работает за линейное время --- выделение за константу и $n$ присваиваний. А если мы хотим инициализацию тоже за константу?

Давайте сдеаем так --- не будем ничего хранить, пока не надо, а когда будет запрос в ранее незанятую ячейку, возвращать ноль.

Перепишем наш Read:

\begin{lstlisting}
Read(i)
    if IsInitialized(i) then
        return A[i]
    else return 0
\end{lstlisting}

\begin{lstlisting}
IsInitialized(i)
    if B[i] > k then 
        return False
    if C[B[i]] = i then
        return True
    else return False
\end{lstlisting}

\begin{lstlisting}
Write(a, i)
    A[i] := a
    if not IsInit(i)
        k := k+1
        B[i] := k
        C[k] := 1
\end{lstlisting}

\begin{lstlisting}
Init(m)
    A := Malloc(n)
    B := Malloc(n)
    C := Malloc(n)
    k := 0
\end{lstlisting}

\subsection{??}

Пусть у нас есть восьмибитное число 0; будем прибавлять к нему единицу. Заметим, что в первый раз понадобится одна операция, потом две, потом одна, потом три\ldots

Получается, что инкремент имеет сложность $O(k)$; логично предположить, что $n$ инкрементов --- $O(nk)$. Это действительно так, но это грубая оценка --- $O(k)$ ведь только в худшем случае.

\begin{lstlisting}
Increment(A)
    for i := 0 to k-1 do
        if A[i] = 0 then
            A[i] := 1
            break
        A[i] = 0
\end{lstlisting}

Заметим, что при $n$ инкрементах последний знак числа меняется $n$ раз, предпоследний --- $\frac{n}{2}$ и так далее.В сумме это составит не более $2n$ изменений ---$O(n)$.

\subsection{Банковский метод}.

$c_i$ --- реальная стоимость

$\hat c_i$ --- учётная стоимость.

$\sum \hat c_i \geqslant \sum c_i$ --- вчёт всегда больше 0.

Если $\hat c_i > c_i$ то разность поступает на счёт, иначе списывается.

При этом учётная стоимость является оценкой сверху для реальности.

В нашей задаче с инкрементом введём такие стоимости:

$\hat c (0\to 1) = 2;\ c (1\to 0) = 0$ 

Учётная стоимость одного инкремента не больше двух.

\subsection{Метод потенциалов}

$D_i$ --- структура данных после $i$ операций.

$\Phi(D_0) = 0$

$\Phi(D_i) \geqslant 0$

$\hat c_i := c_i + \Phi(D_i) - \Phi(D_{i-1})$

$\sum\limits_{i=1}^n\hat c_i = \sum\limits_{i=1}^n c_i +\Phi(D_n) - \Phi(D_0)$

Пусть $\Phi(D_i)$ --- количество единиц.

$\hat c_i = c_i + \Phi(D_i) - \Phi(D_{i-1}) \leqslant t+1+1-t = 2$

\end{document}
