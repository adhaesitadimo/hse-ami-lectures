%% Определение размера листа и шрифта
\documentclass[a4paper,12pt]{article}

%% Работа с русским языком
\usepackage{cmap}					% поиск в PDF
\usepackage{mathtext} 				% русские буквы в формулах
\usepackage[T2A]{fontenc}			% кодировка
\usepackage[utf8]{inputenc}			% кодировка исходного текста
\usepackage[english,russian]{babel}	% локализация и переносы

%% Изменяем размер полей
\usepackage[top=0.5in, bottom=0.75in, left=0.625in, right=0.625in]{geometry}

%% Графика
\usepackage[pdftex]{graphicx}
\graphicspath{{images/}}

%% Различные пакеты для работы с математикой
\usepackage{mathtools}				% Тот же amsmath, только с некоторыми поправками
\usepackage{amssymb}				% Математические символы
\usepackage{amsthm}					% Пакет для написания теорем
\usepackage{amstext}
\usepackage{array}
\usepackage{amsfonts}
\usepackage{icomma}					% "Умная" запятая: $0,2$ --- число, $0, 2$ --- перечисление
\usepackage{mathbbol}				% Для \mathbb со строчными буквами

% Номера формул
\mathtoolsset{showonlyrefs=true} % Показывать номера только у тех формул, на которые есть \eqref{} в тексте.

% Ссылки
\usepackage[colorlinks=true, urlcolor=blue]{hyperref}

% Шрифты
\usepackage{euscript}	 % Шрифт Евклид
\usepackage{mathrsfs}	 % Красивый матшрифт

% Свои команды\textbf{}
\DeclareMathOperator{\sgn}{\mathop{sgn}}

% Перенос знаков в формулах (по Львовскому)
\newcommand*{\hm}[1]{#1\nobreak\discretionary{}
{\hbox{$\mathsurround=0pt #1$}}{}}

% Графики
\usepackage{pgfplots}
%\pgfplotsset{compat=1.12}

% Изменим формат \section и \subsection:
\usepackage{titlesec}
\titleformat{\section}
{\vspace{1cm}\centering\LARGE\bfseries}	% Стиль заголовка
{}										% префикс
{0pt}									% Расстояние между префиксом и заголовком
{} 										% Как отображается префикс
\titleformat{\subsection}				% Аналогично для \subsection
{\Large\bfseries}
{}
{0pt}
{}

% Информация об авторах
\author{Группа лектория ФКН ПМИ 2015-2016 \\
	Ася Иовлева \\
	Ксюша Закирова \\
	Руслан Хайдуров}
\title{Лекции по предмету \\
	\textbf{Линейная алгебра и геометрия}}
\date{2016 год}

\newtheorem*{Def}{Определение}
\newtheorem*{Lemma}{Лемма}
\newtheorem*{Suggestion}{Предложение}
\newtheorem*{Examples}{Пример}
\newtheorem*{Comment}{Замечание}
\newtheorem*{Consequence}{Следствие}
\newtheorem*{Theorem}{Теорема}
\newtheorem*{Statement}{Утверждение}
\newtheorem*{Task}{Упражнение}
\newtheorem*{Designation}{Обозначение}
\newtheorem*{Generalization}{Обобщение}
\newtheorem*{Thedream}{Предел мечтаний}

\renewcommand{\Re}{\mathrm{Re\:}}
\renewcommand{\Im}{\mathrm{Im\:}}
\newcommand{\Arg}{\mathrm{Arg\:}}
\renewcommand{\arg}{\mathrm{arg\:}}
\newcommand{\Mat}{\mathrm{Mat}}
\newcommand{\id}{\mathrm{id}}
\newcommand{\isom}{\xrightarrow{\sim}} 
\newcommand{\leftisom}{\xleftarrow{\sim}}
\newcommand{\Hom}{\mathrm{Hom}}
\newcommand{\Ker}{\mathrm{Ker}\:}
\newcommand{\rk}{\mathrm{rk}\:}

\let\epsilent\varepsilon
\begin{document}

\section*{Лекция 9 от 9.02.2016}

\subsection{Продажа земли}

Предположим, что у вас есть участок земли у берега и вы хотите его продать; при жтом у вас есть несколько покупателей и каждый из них готов отдать некоторую сумму за некоторый фрагмент участка. Как максимизировать выгоду?

Формализуем: у нас есть $n$ предложений и каждое характеерихуется тремя числами:

$s_1, \ldots, s_n$ --- начала

$f_1, \ldots, f_n$ --- концы

$w_1, \ldots, w_n$ --- веса

Пример:

\includegraphics[width=5cm]{09_intervals.jpg}

На выходе мы хотим получить максимальную сумму весов непересекающихся интервалов:

$\max\limits_{T\subseteq \left\{ 1,\ldots, n \right\}; \forall i<j \implies f_i \leqslant s_j \lor f_j \leqslant s_i} \sum\limits_{i\in T}w_i$

Давайте в качестве первого шага отсортируем по правым концам за $O(n\log n)$. Позже будет ясно, зачем.

Пусть $O$ --- оптимальное решение, а OPT$(i)$ --- стоимость оптимального решения для первых $i$ интервалов. А мы хотим вычислить OPT$(n)$.

Например:

\[
    \mathrm{OPT}(6) = \max\begin{cases}
        \mathrm{OPT}(5), 6\not\in O;\\
        2+ \mathrm{OPT}(3), 6\not\in O; \text{(так как все до третьего не пересекаются с шестым)}
    \end{cases}
\]

Пусть $p(j) = \max\left\{ i<j \mid f_{i}\leqslant s_j\right\}$

Эффективное вычисление $p$ остается в качестве упражнения.

Тогда общая формула для OPT такова:
\[
    \mathrm{OPT}(i) = \max\begin{cases}
        \mathrm{OPT}(i-1);\\
        w_i+ \mathrm{OPT}(p(i));
    \end{cases}
\]

\begin{lstlisting}
ComputeOpt(i):
    if i = 0 then return 0
    return max{ComputeOpt(i-1), w_i+ComputeOpt(p(i))}
\end{lstlisting}

Считая, что данные уже отсортированы, получим что сложность равна $O(n)$, но только если мы сохраняем результаты вычислений; иначе мы делаем много лишних вычислений, и время будет таким: $T(n) = T(n-1)+T(n-2)+c$. Очень похоже на числа Фибоначчи, а они растут экспоненциально; это выражение --- тоже.


Значит надо сохранять вычисления в некоторый массив OPT. Инициализируем его так: $\mathrm{OPT} = [0, -1, \ldots, -1]$
\begin{lstlisting}
ComputeOpt(i):
    if OPT[i] < 0 then
        OPT[i] := max{ComputeOpt(i-1), w_i+ComputeOpt(p(i))}
    return OPT[i]
\end{lstlisting}

Другой вариант просто заполнит массив без рекурсии:
\begin{lstlisting}
ComputeOpt(...)
    OPT:= [0, -1, ..., -1]
    for i:= 1 to n do
        OPT[i]:= max{ComputeOpt(i-1), w_i+ComputeOpt(p(i))}
    return OPT[n]
\end{lstlisting}

А теперь попробуем восстановить решение по массиву OPT. Можно его сохранять на каждом шаге, конечно, но это замедлит алгоритм.

\begin{lstlisting}
Find_Solution(OPT):
    T:= {}
    i:= n
    while i > 0
        if OPT[i-1] > w_i + OPT[p(i)] then
            i:=i-1
        else
            T:= T \cup {i}
            i:= p(i)
    return T
\end{lstlisting}

\subsection{В общем о динамическом программировании}
Чем оно отличается от ``Разделяй и властвуй''? А тем, что задачи могут пересекаться. Ведь при использовании классического ``разделяй и властвуй'' мы бы получили экспоненциальное решение. 

Для эффективного использования этого принципа нужно вот что:
\begin{itemize}
    \item Небольшое число задач; например, полиномиальное;
    \item Возможность их упорядочить и выразить решения следующих через предыдущие.
\end{itemize}

\subsection{Задача с прошлой лекции --- выравнивание текста}

Даны длины слов $w_1,\ldots, w_n$ и функция штрафа $c(i, j)$. Мы построили рекурсивное решение с мемоизацией; теперь построим итеративное решение.

OPT$(i)$ --- оптимальное размещение $w_i, \ldots, w_n$.

$\mathrm{OPT}(i) = \min\limits_{i\leqslant j\leqslant n} \left\{ c(i, j)+ \mathrm{OPT}(j+1) \right\}$.

\begin{lstlisting}
ComputeOpt(w_1,..., w_n)
    best = [0]*n
    OPT:= [+\infty]*(n+1)
    OPT[n+1]:= 0
    for i:= n downto 1 do
        for j:= i to n do
            if c(i, j) + OPT[j+1] <= OPT[i] then
                OPT[i]:= c(i, j) _ OPT[j+1]
                best[i]:= j
    i:= 1
    while i < n do
        print best[i]
        i:= best[i]+1
\end{lstlisting}
\end{document}
