% Размер страницы и шрифта
\documentclass[12pt,a4paper]{article}

%% Работа с русским языком
\usepackage{cmap}                   % поиск в PDF
\usepackage{mathtext}               % русские буквы в формулах
\usepackage[T2A]{fontenc}           % кодировка
\usepackage[utf8]{inputenc}         % кодировка исходного текста
\usepackage[english,russian]{babel} % локализация и переносы

%% Изменяем размер полей
\usepackage[top=0.5in, bottom=0.75in, left=0.625in, right=0.625in]{geometry}

%% Различные пакеты для работы с математикой
\usepackage{mathtools}  % Тот же amsmath, только с некоторыми поправками
\usepackage{amssymb}    % Математические символы
\usepackage{amsthm}     % Пакет для написания теорем
\usepackage{amstext}
\usepackage{array}
\usepackage{amsfonts}
\usepackage{icomma}     % "Умная" запятая: $0,2$ --- число, $0, 2$ --- перечисление

%% Графика
\usepackage[pdftex]{graphicx}
\graphicspath{{images/}}

%% Прочие пакеты
\usepackage{listings}               % Пакет для написания кода на каком-то языке программирования
\usepackage{algorithm}              % Пакет для написания алгоритмов
\usepackage[noend]{algpseudocode}   % Подключает псевдокод, отключает end if и иже с ними
\usepackage{indentfirst}            % Начало текста с красной строки
\usepackage[colorlinks=true, urlcolor=blue]{hyperref}   % Ссылки
\usepackage{pgfplots}               % Графики
\pgfplotsset{compat=1.12}
\usepackage{forest}                 % Деревья
\usepackage{titlesec}               % Изменение формата заголовков
\usepackage[normalem]{ulem}         % Для зачёркиваний
\usepackage[autocite=footnote]{biblatex}    % Кавычки для цитат и прочее
\usepackage[makeroom]{cancel}       % И снова зачёркивание (на этот раз косое)

% Изменим формат \section и \subsection:
\titleformat{\section}
	{\vspace{1cm}\centering\LARGE\bfseries} % Стиль заголовка
	{}                                      % префикс
	{0pt}                                   % Расстояние между префиксом и заголовком
	{}                                      % Как отображается префикс
\titleformat{\subsection}                   	% Аналогично для \subsection
	{\Large\bfseries}
	{}
	{0pt}
	{}

% Поправленный вид lstlisting
\lstset { %
    backgroundcolor=\color{black!5}, % set backgroundcolor
    basicstyle=\footnotesize,% basic font setting
}

% Теоремы и утверждения. В комменте указываем номер лекции, в которой это используется.
\newtheorem*{hanoi_recurrent}{Свойство} % Лекция 1
\let\epsilent\varepsilon                % Лекция 8
\DeclareMathOperator{\rk}{rank}         % Лекция 20

% Информация об авторах
\author{Группа лектория ФКН ПМИ 2015-2016 \\
	Никита Попов \\
	Тамерлан Таболов \\
	Лёша Хачиянц}
\title{Лекции по предмету \\
	\textbf{Алгоритмы и структуры данных}}
\date{2016 год}


\begin{document}

\section*{Лекция 5 от 26.01.2016}

\subsection*{Быстрая сортировка. Продолжение}
Говоря об алгоритме быстрой сортировки (\textsc{QSort}), мы рассматривали только случаи, когда все элементы различны. Однако это далеко не всегда так. Если в входном массиве есть равные элементы, то алгоритм может застопориться. Для того, чтобы избежать этого, изменим алгоритм \textsc{Partition}. Попытаемся преобразовывать массив таким образом, чтобы в левой части стояли элементы строго меньшие опорного, в правой --- строго большие, а в середине --- равные ему:
\[\begin{array}{|c|c|c|c|c|}
	\hline
	\phantom{x}  & \dots & x & \dots & \phantom{x}  \\
	\hline
\end{array}
\longrightarrow
\begin{array}{|c|c|c|}
    \hline
    <x&=x&>x\\
    \hline
\end{array}\]

Обозначим за опорный элемент последний. Будем проходиться по массиву от начала до конца, выставляя элементы в нужном порядке (? — ещё не просмотренные элементы):
\[\begin{array}{lllll}
    \hline
   	\multicolumn{1}{|c|}{<x} & \multicolumn{1}{|c|}{>x} & \multicolumn{1}{|c|}{  ?  } & \multicolumn{1}{|c|}{=x} & \multicolumn{1}{|c|}{x}\\
    \hline
    [1, i) & [i, j) & [j, k) & [k, n) & n
\end{array}\]

\begin{algorithm}
\caption{Модифицированный алгоритм \textsc{Partition}}
\begin{algorithmic}[1]
\Function{Partition}{$a$}
\State $i \mathrel{:=} 1$
\State $j \mathrel{:=} 1$
\State $k \mathrel{:=} n - 1$
\While{$j < k$}
	\If{$a[j] = a[n]$}
		\State $k \mathrel{:=} k - 1$
		\State $a[j], a[k] \mathrel{:=} a[k], a[j]$
	\Else \If{$a[j] < a[n]$}
		\State $a[i], a[j] \mathrel{:=} a[j], a[i]$
		\State $j \mathrel{:=} j + 1$
		\State $i \mathrel{:=} i + 1$
	\Else
		\State $j \mathrel{:=} j + 1$
	\EndIf
	\EndIf
\EndWhile
\EndFunction
\end{algorithmic}
\end{algorithm}

Заметим, что $j = k$ (так как алгоритм не закончит работу до тех пор, пока это не станет верно). Тогда на выходе получится массив вида:

\[\begin{array}{lll}
    \hline
    \multicolumn{1}{|c|}{<x} & \multicolumn{1}{|c|}{>x} & \multicolumn{1}{|c|}{=x}\\
    \hline
    [1, i) & [i, j) & [k, n]
\end{array}\]

Остаётся только переставить части массива:

\begin{algorithm}
\begin{algorithmic}[1]
\State  $j \mathrel{:=} n$
\While{$i < k$ and $j \geqslant k$}
	\State $a[i], a[j] \mathrel{:=} a[j], a[i]$
	\State $i \mathrel{:=} i + 1$
	\State $j \mathrel{:=} j - 1$
\EndWhile
\end{algorithmic}
\end{algorithm}
 
\

\textbf{В:} Самая быстрая из наших сортировок --- $O(n \log n)$. А можно ли быстрее?

\textbf{О:} На основе только сравнений --- нет. 

Использовать разобранные нами сортировки можно на любых сущностях, для которых определена операция сравнения.

Предположим теперь, что мы сортируем натуральные числа, не превосходящие некоторого числа $C$.

Создадим массив $b$ размера $C$, заполненный нулями. Будем проходить по исходному массиву $a$ и на каждом шаге будем добавлять 1 к соответствующему элементу массива $b$: 
\[b[a[i]] \mathrel{:=} b[a[i]] + 1\]
Потом, проходя по получившемуся массиву $b$, будем восстанавливать исходный массив уже в отсортированном виде.

Такая сортировка будет работать за $O(n)$, однако, она не универсальна.

Вернёмся к универсальным сортировкам. Рассмотрим дерево для массива $a = [6, 5, 2]$:

\begin{center}
\begin{forest}
for tree={
	%parent anchor=south,
	%child anchor=north,
	if n children=0{
		font=\itshape,
		%tier=terminal,
	}{},
}
[{$\mathbf{a[1] < a[2]}$}
	[{$a[1] < a[3]$}
		[{$a[2] < a[3]$}
			[{$(6, 5, 2)$}
			]
			[{$(5, 6, 2)$}
			]
		]
		[{$(2, 6, 5)$}
		]
	]
	[{$\mathbf{a[2] < a[3]}$}
		[{$a[1] < a[3]$}
			[{$(5, 6, 2)$}
			]
			[{$(5, 2, 6)$}
			]
		]
		[{$\mathbf{(2, 6, 5)}$}
		]
	]
]
\end{forest}
\end{center}

Подобное дерево можно составить для любого детерминированного\footnote{Детерминированный алгоритм — алгоритмический процесс, который выдаёт предопределённый результат для заданных входных данных. Например, \textsc{QSort}, выбирающий опорный элемент случайным образом, не является детерминированным.} алгоритма сортировки, зафиксировав $n$. Сложность алгоритма будет являться высота $h$ дерева. Посчитаем это $h$:
\begin{itemize}
	\item[$\blacktriangleright$] Так как алгоритм должен работать на любой перестановке из $n$ элементов, то у дерева не может быть меньше, чем $n!$ листов.
	\item[$\blacktriangleright$] Так как сравнение --- бинарная операция, то у каждой вершины не более двух потомков. Тогда в дереве не может быть больше, чем $2^h$ листьев.
	\item[$\blacktriangleright$] Тогда $2^h \geqslant n! \iff h \geqslant \log_2 n!$.
	Заметим, что:
	\[n! = 1 \cdot 2 \cdot \ldots \cdot \left\lfloor\frac{n}{2}\right\rfloor \cdot \underbrace{ \left(\left\lfloor\frac{n}{2}\right\rfloor + 1\right) \cdot \ldots (n - 1) \cdot n}_\text{каждый из $\frac{n}{2}$ элементов не меньше $\frac{n}{2}$} \geqslant \left(\frac{n}{2}\right)^{\frac{n}{2}} \]
	Тогда $h \geqslant \log_2 \left(\frac{n}{2}\right)^{\frac{n}{2}} = \frac{n}{2} \log_2 \frac{n}{2} = \Omega(n \log n)$. Из этого следует, что отсортировать произвольный массив с помощью только сравнений меньше, чем за $\Omega(n \log n)$ операций, невозможно.
\end{itemize}

\subsection*{Поиск медианы}

Медиана --- такой элемент массива, что не меньше половины элементов меньше неё, и не меньше половины --- больше.

Для отсортированного массива размера $n$ медиана будет  находиться под номером $\dfrac{n + 1}{2}$ для нечётных $n$ и $\dfrac{n}{2}$ для чётных $n$.
Пример: для массива (8, 1, 3, 5, 6, 9) медианой будет являться~5.

Как же найти медиану?
Очевидно, что можно отсортировать и взять средний --- $\Theta(n)$.

А можно ли найти медиану ли за линейное время?
Можно.
Напишем алгоритм, находящий элемент, стоящий на k-ом месте в массиве, получающемся из входного после сортировки.
Это называется поиском $k$-ой порядковой статистики.
Составим этот алгоритм, немного модифицировав QSort:

\begin{algorithm}
\caption{Поиск $k$-ой порядковой статистики}
\begin{algorithmic}[1]
\Function{Select}{$a, k$}
	\State choose pivot $a[p]$
	\State $i \mathrel{:=} \textsc{Partition}(a, p)$
	\If{$i \mathrel{:=} k$}
		\State return $a[i]$	
	\EndIf
	\If{$i > k$}
		\State return $\textsc{Select}(a[1 \ldots i - 1], k)$
	\Else
		\State return $\textsc{Select}(a[i + 1 \ldots n], k - i)$
	\EndIf
\EndFunction
\end{algorithmic}
\end{algorithm}

Как и в быстрой сортировке, неправильно выбранный опорный элемент портит скорость до $n^2$. Будем выбирать опорный элемент случайным образом. Попробуем посчитать время работы в среднем случае.

$j$-подзадача размера $n'$. $\left( \frac{3}{4} \right)^{j+1}n < n' \leqslant \left( \frac{3}{4} \right)^{j}n$

Как и в QSort, в среднем мы потратим две попытки на переход к следующему $j$.

Максимальное $j$ --- $O(\log_\frac{4}{3} n)$

$T(n) \leqslant \sum\limits_{j=0}^{\log_{\frac{4}{3}}n} 2\cdot c\cdot \left( \frac{3}{4} \right)^jn = 2cn\sum\limits_{j=0}^{\log_{\frac{4}{3}}n}\left( \frac{3}{4} \right)^j \leqslant 2cn$

Время работы алгоритма в худшем случае всё ещё $O(n^2)$.
Худший случай — когда на каждом шаге мы отщеплем всего один элемент.
Для достижения лучшего случая, на каждом шаге нужно выбирать в качестве опорного элемента медиану.

\subsection*{Медиана медиан}
Попробуем несколько модифицировать наш алгоритм.
Разобьём входной массив на группы по 5 элементов.
Отсортируем каждую такую группу.
Так как размер каждой группы зафиксирован, время сортировки не зависит от $n$.
Зависит только количество сортировок.
Возьмём медиану в каждой группе и применим алгоритм нахождения медианы к получившемуся массиву медиан.
Выберем её в качестве опорного элемента.

\begin{algorithm}
\caption{Поиск $k$-ой порядковой статистики 2}
\begin{algorithmic}[1]
\Function{Select}{$a, k$}
	\State Divide a into groups of 5
	\State Choose medians $m_1,\ldots m_\frac{n}{5}$
	\State $x = \textsc{Select}([m_1,\ldots, m_\frac{n}{5}], \frac{n}{10})$
	\State choose $x$ as pivot $a[p]$
	\State $i \mathrel{:=} \textsc{Partition}(a, p)$
	\If{$i \mathrel{:=} k$}
		\State return $a[i]$	
	\EndIf	
	\If{$i > k$}
		\State return $\textsc{Select}(a[1 \ldots i - 1], k)$
	\Else
		\State return $\textsc{Select}(a[i + 1 \ldots n], k - i)$
	\EndIf
\EndFunction
\end{algorithmic}
\end{algorithm}

$T(n) \leqslant cn + T\left(\frac{n}{5}\right) + T\left( \frac{7}{10}n \right)$

$T(n) \leqslant ln$ для некоторого $l$

$T(n) \leqslant cn + T(\frac{n}{5}) + T(\frac{7}{10}) \leqslant cn + \frac{ln}{5} + \frac{7}{10}ln$
\end{document}
