%% Определение размера листа и шрифта
\documentclass[a4paper,12pt]{article}

%% Работа с русским языком
\usepackage{cmap}					% поиск в PDF
\usepackage{mathtext} 				% русские буквы в формулах
\usepackage[T2A]{fontenc}			% кодировка
\usepackage[utf8]{inputenc}			% кодировка исходного текста
\usepackage[english,russian]{babel}	% локализация и переносы

%% Изменяем размер полей
\usepackage[top=0.5in, bottom=0.75in, left=0.625in, right=0.625in]{geometry}

%% Графика
\usepackage[pdftex]{graphicx}
\graphicspath{{images/}}

%% Различные пакеты для работы с математикой
\usepackage{mathtools}				% Тот же amsmath, только с некоторыми поправками
\usepackage{amssymb}				% Математические символы
\usepackage{amsthm}					% Пакет для написания теорем
\usepackage{amstext}
\usepackage{array}
\usepackage{amsfonts}
\usepackage{icomma}					% "Умная" запятая: $0,2$ --- число, $0, 2$ --- перечисление
\usepackage{mathbbol}				% Для \mathbb со строчными буквами

% Номера формул
\mathtoolsset{showonlyrefs=true} % Показывать номера только у тех формул, на которые есть \eqref{} в тексте.

% Ссылки
\usepackage[colorlinks=true, urlcolor=blue]{hyperref}

% Шрифты
\usepackage{euscript}	 % Шрифт Евклид
\usepackage{mathrsfs}	 % Красивый матшрифт

% Свои команды\textbf{}
\DeclareMathOperator{\sgn}{\mathop{sgn}}

% Перенос знаков в формулах (по Львовскому)
\newcommand*{\hm}[1]{#1\nobreak\discretionary{}
{\hbox{$\mathsurround=0pt #1$}}{}}

% Графики
\usepackage{pgfplots}
%\pgfplotsset{compat=1.12}

% Изменим формат \section и \subsection:
\usepackage{titlesec}
\titleformat{\section}
{\vspace{1cm}\centering\LARGE\bfseries}	% Стиль заголовка
{}										% префикс
{0pt}									% Расстояние между префиксом и заголовком
{} 										% Как отображается префикс
\titleformat{\subsection}				% Аналогично для \subsection
{\Large\bfseries}
{}
{0pt}
{}

% Информация об авторах
\author{Группа лектория ФКН ПМИ 2015-2016 \\
	Ася Иовлева \\
	Ксюша Закирова \\
	Руслан Хайдуров}
\title{Лекции по предмету \\
	\textbf{Линейная алгебра и геометрия}}
\date{2016 год}

\newtheorem*{Def}{Определение}
\newtheorem*{Lemma}{Лемма}
\newtheorem*{Suggestion}{Предложение}
\newtheorem*{Examples}{Пример}
\newtheorem*{Comment}{Замечание}
\newtheorem*{Consequence}{Следствие}
\newtheorem*{Theorem}{Теорема}
\newtheorem*{Statement}{Утверждение}
\newtheorem*{Task}{Упражнение}
\newtheorem*{Designation}{Обозначение}
\newtheorem*{Generalization}{Обобщение}
\newtheorem*{Thedream}{Предел мечтаний}

\renewcommand{\Re}{\mathrm{Re\:}}
\renewcommand{\Im}{\mathrm{Im\:}}
\newcommand{\Arg}{\mathrm{Arg\:}}
\renewcommand{\arg}{\mathrm{arg\:}}
\newcommand{\Mat}{\mathrm{Mat}}
\newcommand{\id}{\mathrm{id}}
\newcommand{\isom}{\xrightarrow{\sim}} 
\newcommand{\leftisom}{\xleftarrow{\sim}}
\newcommand{\Hom}{\mathrm{Hom}}
\newcommand{\Ker}{\mathrm{Ker}\:}
\newcommand{\rk}{\mathrm{rk}\:}

\let\epsilent\varepsilon
\begin{document}

\section*{Лекция 8 от 4.02.2016}

\subsection{Возведение в степень}
Пусть $x$ и $y$ содержат по $n$ цифр.

Можно ли за полиномиальное время возвести число $x$ в степень $y$?

Если мы тривиально перемножим $y$ чисел $x$, несложно показать, что сложность алгоритма будет $O(2^n)$ (где $n$ --- число цифр в числе).

Заметим, что число $x^y$ содержит $n\cdot10^n$ цифр.

Получается, что один только размер результата экспоненциален, то есть полиномиальной сложности не хватит даже на вывод результата.

А если по модулю?

Вход: $x,\ y,\ p$ (по $n$ цифр).

Выход: $x^y \pmod{p}$.

$x, x^2\pmod{p}, x^3\pmod{p}\ldots$.

Попробуем \emph{быстрое возведение в степень}.

\begin{lstlisting}
Power(x, y, p)
    if y = 0 then
        return 1
    t:= Power(x, floor(y/2), p)
    if y is even then
        return t^2 mod p
    else
        return x*t^2 mod p
\end{lstlisting}

Глубина рекурсии --- $O(\log y) = O(n)$.

Или вот так:

Пример: $x=4,\ y=5$.

$x^1,\ x^2,\ x^4 \to x^5 = x^1\cdot x^4$

\subsection{Обратная задача}

Вход: $x,\ z,\ p$ (по $n$ цифр).

Выход: $y$ такой, что $x^y = z \pmod{p}$.

Такая задача пока не решена за полиномиальное время, но и невозможность этого тоже не доказана. Это всё, вообще говоря, висит на известной проблеме $P\mathrel{\overset{?}{=}} NP$ и подробнее мы об этом поговорим ближе к концу курса.

\subsection{Обработка текста}
Предположим, у нас есть $n$ слов, и эти слова мы хотим разместить на странице (порядок, разумеется, не меняя --- это же, в конце концов, текст). При этом, шрифт моноширинный, а ширина строки ограничена. Что мы хотим --- разместить текст так, чтобы он был выровнен по обоим краям. При этом хотелось бы, чтобы пробелы были примерно одинаковы по ширине.

Введём такую ??? (меру? хз): $\epsilent(i, j) = L-\sum\limits_{t=i}^j|w_t|-(j-i)$ --- число дополнительных пробелов в строке с $i$-го по $j$-ое слово.

Также введём $c(i, j)$ --- стоимость размещения.

\[
    c(i, j) = \begin{cases}
        +\infty, \epsilent(i, j) < 0\\
        \left( \frac{\epsilent(i, j)}{j-i} \right)^3, \epsilent(i, j) \geqslant 0\\
    \end{cases}
\]

И как это решать? Можно попробовать жадным алгоритмом --- просто ``впихивать'' слова в строку, пока влезают. Он тут не работает, так как он вообще не учитывает стоимость.

Попробуем наш извечный ``разделяй и властвуй''. Базовый случай --- слова помещаются в одну строку, а если не помещаются --- переносим и повторяем. Но тут тоже не учитывается стоимость, так что вряд ли будет сильно лучше.

Вход: $w_1, \ldots, w_n; c(i, j)$.

Выход: $j_0, \ldots, j_{l+1}$, такие что $j_0 = 1,\ j_{l+1} = n,\ \sum c(j_i, j_{i+1})$ минимальна.

Сколько всего таких наборов? Мест, где в принципе может оказаться разрып строки --- $n-1$, в каждом можно поставить или не поставить --- итого $2^{n-1}$ разбиений.

Пусть $OPT(j)$ --- стоимость оптимального размещения слов с $j$-ого по $n$-ное. Наша задача --- вычислить $OPT(1)$. А как?

\[
    OPT(1) = \min\limits_{i\leqslant n}\{c(1, i)+OPT(i+1)\}
\]

\begin{lstlisting}
OPT(j):
    if j = n+1 then return 0
    f:= +inf
    for i:= j to n do
     f:= min(f, c(i, j)+OPT(i+1))
\end{lstlisting}

$(*)\ OPT(j) =\begin{cases}
    0, j>n\\
    \min\limits_{i = j\ldots n}\left\{ c(j, i) + OPT(i+1) \right\}
\end{cases}$

А сложность? Построив дерево, заметим, что $OPT(3)$ вычисляется два раза; $OPT(4)$ -- три раза и так далее.

Будем сохранять результаты:

\begin{lstlisting}
OPT_cache(j):
    if M[j] != NULL then
    else
        M[j] = OPT(j)
    return M[j]
\end{lstlisting}

Такая методика называется \emph{динамическим программированием}.

Основная идея --- каждая задача зависит от полиномиального числа других задач.

\end{document}
