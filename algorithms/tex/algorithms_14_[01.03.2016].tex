% Размер страницы и шрифта
\documentclass[12pt,a4paper]{article}

%% Работа с русским языком
\usepackage{cmap}                   % поиск в PDF
\usepackage{mathtext}               % русские буквы в формулах
\usepackage[T2A]{fontenc}           % кодировка
\usepackage[utf8]{inputenc}         % кодировка исходного текста
\usepackage[english,russian]{babel} % локализация и переносы

%% Изменяем размер полей
\usepackage[top=0.5in, bottom=0.75in, left=0.625in, right=0.625in]{geometry}

%% Различные пакеты для работы с математикой
\usepackage{mathtools}  % Тот же amsmath, только с некоторыми поправками
\usepackage{amssymb}    % Математические символы
\usepackage{amsthm}     % Пакет для написания теорем
\usepackage{amstext}
\usepackage{array}
\usepackage{amsfonts}
\usepackage{icomma}     % "Умная" запятая: $0,2$ --- число, $0, 2$ --- перечисление

%% Графика
\usepackage[pdftex]{graphicx}
\graphicspath{{images/}}

%% Прочие пакеты
\usepackage{listings}               % Пакет для написания кода на каком-то языке программирования
\usepackage{algorithm}              % Пакет для написания алгоритмов
\usepackage[noend]{algpseudocode}   % Подключает псевдокод, отключает end if и иже с ними
\usepackage{indentfirst}            % Начало текста с красной строки
\usepackage[colorlinks=true, urlcolor=blue]{hyperref}   % Ссылки
\usepackage{pgfplots}               % Графики
\pgfplotsset{compat=1.12}
\usepackage{forest}                 % Деревья
\usepackage{titlesec}               % Изменение формата заголовков
\usepackage[normalem]{ulem}         % Для зачёркиваний
\usepackage[autocite=footnote]{biblatex}    % Кавычки для цитат и прочее
\usepackage[makeroom]{cancel}       % И снова зачёркивание (на этот раз косое)

% Изменим формат \section и \subsection:
\titleformat{\section}
	{\vspace{1cm}\centering\LARGE\bfseries} % Стиль заголовка
	{}                                      % префикс
	{0pt}                                   % Расстояние между префиксом и заголовком
	{}                                      % Как отображается префикс
\titleformat{\subsection}                   	% Аналогично для \subsection
	{\Large\bfseries}
	{}
	{0pt}
	{}

% Поправленный вид lstlisting
\lstset { %
    backgroundcolor=\color{black!5}, % set backgroundcolor
    basicstyle=\footnotesize,% basic font setting
}

% Теоремы и утверждения. В комменте указываем номер лекции, в которой это используется.
\newtheorem*{hanoi_recurrent}{Свойство} % Лекция 1
\let\epsilent\varepsilon                % Лекция 8
\DeclareMathOperator{\rk}{rank}         % Лекция 20

% Информация об авторах
\author{Группа лектория ФКН ПМИ 2015-2016 \\
	Никита Попов \\
	Тамерлан Таболов \\
	Лёша Хачиянц}
\title{Лекции по предмету \\
	\textbf{Алгоритмы и структуры данных}}
\date{2016 год}


\begin{document}

\section*{Лекция 14 от 01.03.2016}

\subsection{Поиск компонент связности в неориентированном графе}
(третий аргумент DFS --- то,  чем помечаются вершины)
\begin{lstlisting}
i := 1
for v in V do
    if v is not marked then
        DFS(G, v, i)
        i := i + 1
\end{lstlisting}

Сколько времени, если в графе $m$ вершин и $n$ рёбер? $O(m_n)$. Почему? Потому что в DFS мы проходим по каждому ребру в компоненте связности не более двух раз.

А для ориентированного графа?

Проверка сильной связности в ориентированном графе. Очевидно, это сложнее, потому что, если мы рассмотрим граф из двух вершин и одного ребра, то из одной DFS до другой доберётся, а в обратную сторону --- уже нет.

\begin{lstlisting}
DFS(G, s, 1)
if exists u in V: u is not marked then
    return false
return DFS(Grev, s, 2)
\end{lstlisting}

Если $G = (V, E)$, то $G^{rev} = (V, E^{rev})$

Вот как мы определяем $DFS$, кстати:

\begin{lstlisting}
DFS(G, s, i)
    mark s with i
    for (s, v) in E do
        if v is not marked
            DFS(G, v, i)
    t := t+1
    f[s] = t
\end{lstlisting}

Поиск компонент сильной связности:

\begin{lstlisting}
Explore(Grev)
unmark all
Explore(G)  // for v in V v poryadke ubyvuaniya f[v]
\end{lstlisting}

Пусть $C_1$ и $C_2$ --- две компоненты сильной связности; $i\in C_1,\ j\in C_2,\ (i,j)\in E$. Тогда $\max\limits_{u\in C_1} f[u] < \max\limits_{v\in C_2} f[v]$

Каждый вызов DFS внутри Explore(G) обходит некоторую компоненту сильной связности.

S --- множество вершин, обойдённых в Explore(G).

Базис: $S = \varnothing$.
Шаг: Пусть $S = C_1 \cup\ldots\cup C_k$, где $C_i$ --- компоненты сильной связности.

Пусть $C_v$ --- к.с.с., содержащая $v$ и $S_v$ --- множество вершин, обойдённых в DFS(G, v, i). Тогда $C_v \subseteq S_v$; $u\not\in C_v \implies \forall w \in C_u: f[v] < f[w]$.

\subsection{Матрицы смежности}
Пусть $A$ --- матрица смежности $G$; тогда $(A^t)_{ij}$ --- число путей из $i$ в $j$ длины $t$.

Существует ли путь из $i$ в $j$? Возведём $E+A$ в степень $n-1$ и проверим, что $ij$-ый элемент больше нуля. Используя быстрое возведение в степень можно обойтись $\lceil\log_2n\rceil$ перемножениями матриц.

\begin{lstlisting}
B := E + A
for x := 1 to log2 n do
    b_ij := \Lor (b_ik\land b_kj)
\end{lstlisting}

\subsection{Кратчайшие пути}
Вход: $G, W$;
Выход: $B$ такое, что $B_{ij}$ --- длина кратччайшего пути из $i$ в $j$.

Алгоритм Флойда-Уоршелла:

\begin{lstlisting}
B := W
for k in V do
    for i in V do
        for j in V do
        B_{ij}:= min(B_{ij}, B_{ik} + B_{kj})
\end{lstlisting}
\end{document}
