% Размер страницы и шрифта
\documentclass[12pt,a4paper]{article}

%% Работа с русским языком
\usepackage{cmap}                   % поиск в PDF
\usepackage{mathtext}               % русские буквы в формулах
\usepackage[T2A]{fontenc}           % кодировка
\usepackage[utf8]{inputenc}         % кодировка исходного текста
\usepackage[english,russian]{babel} % локализация и переносы

%% Изменяем размер полей
\usepackage[top=0.5in, bottom=0.75in, left=0.625in, right=0.625in]{geometry}

%% Различные пакеты для работы с математикой
\usepackage{mathtools}  % Тот же amsmath, только с некоторыми поправками
\usepackage{amssymb}    % Математические символы
\usepackage{amsthm}     % Пакет для написания теорем
\usepackage{amstext}
\usepackage{array}
\usepackage{amsfonts}
\usepackage{icomma}     % "Умная" запятая: $0,2$ --- число, $0, 2$ --- перечисление

%% Графика
\usepackage[pdftex]{graphicx}
\graphicspath{{images/}}

%% Прочие пакеты
\usepackage{listings}               % Пакет для написания кода на каком-то языке программирования
\usepackage{algorithm}              % Пакет для написания алгоритмов
\usepackage[noend]{algpseudocode}   % Подключает псевдокод, отключает end if и иже с ними
\usepackage{indentfirst}            % Начало текста с красной строки
\usepackage[colorlinks=true, urlcolor=blue]{hyperref}   % Ссылки
\usepackage{pgfplots}               % Графики
\pgfplotsset{compat=1.12}
\usepackage{forest}                 % Деревья
\usepackage{titlesec}               % Изменение формата заголовков
\usepackage[normalem]{ulem}         % Для зачёркиваний
\usepackage[autocite=footnote]{biblatex}    % Кавычки для цитат и прочее
\usepackage[makeroom]{cancel}       % И снова зачёркивание (на этот раз косое)

% Изменим формат \section и \subsection:
\titleformat{\section}
	{\vspace{1cm}\centering\LARGE\bfseries} % Стиль заголовка
	{}                                      % префикс
	{0pt}                                   % Расстояние между префиксом и заголовком
	{}                                      % Как отображается префикс
\titleformat{\subsection}                   	% Аналогично для \subsection
	{\Large\bfseries}
	{}
	{0pt}
	{}

% Поправленный вид lstlisting
\lstset { %
    backgroundcolor=\color{black!5}, % set backgroundcolor
    basicstyle=\footnotesize,% basic font setting
}

% Теоремы и утверждения. В комменте указываем номер лекции, в которой это используется.
\newtheorem*{hanoi_recurrent}{Свойство} % Лекция 1
\let\epsilent\varepsilon                % Лекция 8
\DeclareMathOperator{\rk}{rank}         % Лекция 20

% Информация об авторах
\author{Группа лектория ФКН ПМИ 2015-2016 \\
	Никита Попов \\
	Тамерлан Таболов \\
	Лёша Хачиянц}
\title{Лекции по предмету \\
	\textbf{Алгоритмы и структуры данных}}
\date{2016 год}


\begin{document}

\section{Лекция 25 от 21.03.2016}

\subsection*{Жорданова нормальная форма}
Пусть $V$ --- векторное пространство, $\phi$ --- линейный оператор.

\begin{Theorem}[Жорданова нормальная форма линейного оператора]
Пусть $\chi_\phi(t)$ разлагается на линейные множители. Тогда существует базис $\e$ в $V$ такой, что 
\begin{gather*}
A(\phi, \e) = 
\begin{pmatrix*}
J_{\mu_1}^{n_1} & 0 & \ldots & 0 \\
0 & J_{\mu_2}^{n_2} & \ldots & 0 \\
\vdots & \vdots & \ddots & \vdots \\
0 & 0 & \ldots & J_{\mu_p}^{n_p}
\end{pmatrix*} \quad (*)
\end{gather*}
Кроме того, матрица $(*)$ определена однозначно с точностью до перестановок жордановых клеток.
\end{Theorem}

\begin{Def}
Матрица $(*)$ называется жордановой нормальной формой линейного оператора.
\end{Def}

\begin{Consequence}
В векторном пространстве над полем комплексных чисел для любого линейного оператора существует жорданова нормальная форма.
\end{Consequence}

\textbf{Схема построения:}
\begin{itemize}
\item[Шаг 1:] Разложим характеристический многочлен: $\chi_\phi(t) = (t - \l_1)^{k_1}\ldots(t-\l_s)^{k_s}$, где $\l_i \neq \l_j$ при $i \neq j$. Тогда, по доказанной на прошлой лекции теореме, $V = \bigoplus_{i = 1}^sV^{\l_i}(\phi)$, причем $\dim V^{\l_i}(\phi) = k_i$. 

Введем отображение $\psi_i = \limref{\phi}{V^{\l_i}(\phi)} \in L(V^{\l_i}(\phi))$. Тогда $\chi_{\psi_i}(t) = (t - \l_i)^{k_i}$. Также введем $\e_i$~--- базис $V^{\l_i}(\phi)$. Пусть $\e = \e_1 \cup \ldots \cup \e_s$.

Тогда:
\begin{gather*}
A(\phi, \e) = 
\begin{pmatrix*}
A_1 & 0 & \ldots & 0 \\
0 & A_2 & \ldots & 0 \\
\vdots & \vdots & \ddots & \vdots \\
0 & 0 & \ldots & A_s
\end{pmatrix*}, \quad \text{где } A_i = A(\psi_i, \e_i) \in M_{k_i}.
\end{gather*}

\item[Шаг 2:] Для любого $i$ можно выбрать базис $\e_i$ так, чтобы
\begin{gather*}
A_i = 
\begin{pmatrix*}
J_{\l_i}^{m_{i1}} & 0 & \ldots & 0 \\
0 & J_{\l_i}^{m_{i2}} & \ldots & 0 \\
\vdots & \vdots & \ddots & \vdots \\
0 & 0 & \ldots & J_{\l_i}^{m_{iq}}
\end{pmatrix*}, \quad m_{i1} + \ldots + m_{iq} = k_i
\end{gather*}
Обратите внимание, что здесь все жордановы клетки отвечают одному значению $\l_i$, но при этом матрица $A_i$ целиком жорданову клетку не образует, так как линия единиц над диагональю из $\l$ разрывна там, где состыковываются две клетки:
\begin{gather*}
\begin{pmatrix*}
\begin{array}{c|c}
\begin{matrix}
\l_i & 1 \\
&  \ddots & \ddots\\
&  & \l_i & 1 \\
&  &  & \l_i
\end{matrix} & 
\begin{matrix}
\\
\\
\\
0&&&&&&&
\end{matrix}
\\ \hline & 
\begin{matrix}
\l_i & 1 \\
&  \ddots & \ddots\\
&  & \l_i & 1 \\
&  &  & \l_i
\end{matrix}
\end{array}
\end{pmatrix*}
\end{gather*}

Тогда жорданова нормальная форма матрицы $A(\phi, \e)$ составляется из таких матриц $A_i$: 
\setcounter{MaxMatrixCols}{13}
\begin{gather*}
A(\phi, \e) = 
\begin{pmatrix*}
J_{\l_1}^{m_{11}} \\
& J_{\l_1}^{m_{12}} \\
&&\ddots\\
&&& J_{\l_2}^{m_{21}} \\
&&&& \ddots \\
&&&&& J_{\l_s}^{m_{s1}} \\
&&&&&& \ddots \\
&&&&&&& J_{\l_s}^{m_{sk_i}}
\end{pmatrix*}
\end{gather*}

\item[Шаг 3:] Осталось только заметить, что для любого $i = 1, \ldots, s$  число и порядок жордановых клеток однозначно определены из последовательности чисел:
\begin{align*}
&\dim \ker (\psi_i - \l_i \id) \\
&\dim \ker (\psi_i - \l_i \id) ^ 2 \\
&\ldots \\
&\dim \ker (\psi_i - \l_i \id)^{k_i}
\end{align*}

Откуда и следует однозначность представления в виде жордановой нормальной формы (с точностью до перестановки жордановых клеток).
\end{itemize}

\subsection*{Линейные функции на векторном пространстве}
Начнем с примера. Рассмотрим функцию $f \colon \R^n \rightarrow \R$.

Пусть $x_0 \in \R^n$ и $y = \begin{pmatrix}y_1 \\ \vdots \\ y_n\end{pmatrix} \in\R^n$ --- приращение, то есть $x = x_0 + y$. Если функция достаточно хорошая, то есть дважды дифференцируема в точке $x$, то
\begin{gather*}
f(x) = f(x_0) + a_1y_1 + \ldots + a_ny_n + b_{11}y_1^2 + \ldots + b_{ij}y_iy_j +\ldots + b_{nn}y_n^2 + \overline{o}(|y|^2).
\end{gather*}

Сумма $a_1y_1 + \ldots + a_ny_n$ называется линейной формой, а сумма $b_{11}y_1^2 + \ldots \hm+ b_{ij}y_iy_j + \ldots + b_{nn}y_n^2$~--- квадратичной формой.

Теперь дадим строгое определение:
\begin{Def}
Линейной функцией (формой, функционалом) на векторном пространстве $V$ называется всякое линейное отображение $\sigma \colon V \rightarrow F$. 

Обозначение: $V^* = \Hom(V, F)$.
\end{Def}

В этом определении $F$ фактически рассматривается как одномерное векторного пространство.

\begin{Comment}
Функционалом принято называть, когда векторное пространство состоит из функций.
\end{Comment}

\begin{Examples}\
\begin{enumerate}
\item $\alpha \colon \R^n \rightarrow \R;\ \phi(v) = \langle v, e \rangle$ --- скалярное произведение с некоторым фиксированным $e$.
\item $\alpha \colon \mathcal{F}(X, F) \rightarrow F;\ \alpha(f) = f(x_0)$. Здесь $\mathcal{F}(X, F) = \{f \colon X \rightarrow F  \}$.
\item $\alpha \colon C[a, b] \rightarrow \R;\ \alpha(f) = \int_a^b f(x) dx$.
\item $\alpha \colon M_n(F) \rightarrow F;\ \alpha(X) = \mathrm{tr}A$.
\end{enumerate}
\end{Examples}

\begin{Def}
Пространство $V^*$ называется сопряженным (двойственным) к $V$.
\end{Def}

Пусть $\e = (e_1, \ldots, e_n)$ --- базис $V$. Тогда он определяет изоморфизм $\phi \colon V^* \to \Mat_{1\times n}$, \\$\alpha \mapsto (\alpha_1, \ldots, \alpha_n)$, где $\alpha_i = \phi(e_i)$ и $\alpha$ --- линейная функция. При этом, если $x = x_1e_1 + \ldots + x_ne_n$, то $\alpha(x) = (\alpha_1, \ldots, \alpha_n)\begin{pmatrix}x_1\\ \vdots \\ x_n\end{pmatrix}$.

\begin{Consequence}
$\dim V^* = n$.
\end{Consequence}

Пусть $\e = (e_1, \ldots, e_n)$ --- базис $V$. Рассмотрим линейные функции $\eps_1, \ldots, \eps_n$ такие, что $\eps_i(e_j) =~\delta_{ij}$, где $\delta_{ij} =
\begin{cases}
1, & i = j \\
0, & i \neq j
\end{cases}
$ --- символ Кронекера. То есть $\eps_i = (\delta_{i1}, \ldots, \delta_{ii}, \ldots, \delta_{in}) = (0, \ldots, 1, \ldots, 0)$.

\begin{Suggestion}
$(\eps_1, \ldots, \eps_n)$ --- базис в $V^*$.
\end{Suggestion}

\begin{proof}
Возьмем любое $\alpha \in V^*$. Положим $a_i = \alpha(e_i)$. Тогда $\alpha = a_1\eps_1 + \ldots + a_n\eps_n$. То есть мы получили, что через $(\eps_1, \ldots, \eps_n)$ действительно можно выразить любое $\alpha$.

Теперь покажем, что $\eps_1, \ldots, \eps_n$ --- линейно независимы. Пусть $a_1\eps_1 + \ldots + a_n\eps_n = 0,\ a_i \in F$. Применив эту функцию к $e_i$, получим, что $a_1\eps_1(e_1) + \ldots + a_n\eps_n(e_i) = 0$. Отсюда следует, что $a_i = 0$, а все остальные $a_j$, при $j \neq i$, равны нулю в силу определения $\eps_j$. Итого, $a_1 = \ldots \hm= a_n = 0$, что и доказывает линейную независимость.
\end{proof}

\begin{Def}
Базис $(\eps_1, \ldots, \eps_n)$ называется сопряженным к $\e$ базисом.
\end{Def}

\begin{Task}
Всякий базис $V^*$ сопряжен некоторому базису $V$.
\end{Task}

\subsection*{Билинейные функции на векторном пространстве}

\begin{Def}
Билинейной функцией (формой) на векторном пространстве $V$ называется всякое билинейное отображение $\beta \colon V \times V \rightarrow F$. То есть это отображение, линейное по каждому аргументу:
\begin{enumerate}
\item $\beta(x_1 + x_2, y) = \beta(x_1, y) + \beta(x_2, y)$; 
\item $\beta(\l x, y) = \l\beta(x, y)$;
\item $\beta(x, y_1 + y_2) = \beta(x, y_1) + \beta(x, y_2)$;
\item $\beta(x, \l y) = \l\beta(x, y)$.
\end{enumerate}
\end{Def}

\begin{Examples}\
\begin{enumerate}
\item $V = \R^n,\ \beta(x, y) = \langle x, y \rangle$ --- скалярное произведение. 
\item $V = \R^2,\ \beta(x, y) = \begin{vmatrix}x_1 & y_1 \\ x_2 & y_2\end{vmatrix}$.
\item $V = C[a, b],\ \beta(f, g) = \int_a^bf(x)g(x)dx$.
\end{enumerate}
\end{Examples}

\end{document}
