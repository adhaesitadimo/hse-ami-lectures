% Размер страницы и шрифта
\documentclass[12pt,a4paper]{article}

%% Работа с русским языком
\usepackage{cmap}                   % поиск в PDF
\usepackage{mathtext}               % русские буквы в формулах
\usepackage[T2A]{fontenc}           % кодировка
\usepackage[utf8]{inputenc}         % кодировка исходного текста
\usepackage[english,russian]{babel} % локализация и переносы

%% Изменяем размер полей
\usepackage[top=0.5in, bottom=0.75in, left=0.625in, right=0.625in]{geometry}

%% Различные пакеты для работы с математикой
\usepackage{mathtools}  % Тот же amsmath, только с некоторыми поправками
\usepackage{amssymb}    % Математические символы
\usepackage{amsthm}     % Пакет для написания теорем
\usepackage{amstext}
\usepackage{array}
\usepackage{amsfonts}
\usepackage{icomma}     % "Умная" запятая: $0,2$ --- число, $0, 2$ --- перечисление

%% Графика
\usepackage[pdftex]{graphicx}
\graphicspath{{images/}}

%% Прочие пакеты
\usepackage{listings}               % Пакет для написания кода на каком-то языке программирования
\usepackage{algorithm}              % Пакет для написания алгоритмов
\usepackage[noend]{algpseudocode}   % Подключает псевдокод, отключает end if и иже с ними
\usepackage{indentfirst}            % Начало текста с красной строки
\usepackage[colorlinks=true, urlcolor=blue]{hyperref}   % Ссылки
\usepackage{pgfplots}               % Графики
\pgfplotsset{compat=1.12}
\usepackage{forest}                 % Деревья
\usepackage{titlesec}               % Изменение формата заголовков
\usepackage[normalem]{ulem}         % Для зачёркиваний
\usepackage[autocite=footnote]{biblatex}    % Кавычки для цитат и прочее
\usepackage[makeroom]{cancel}       % И снова зачёркивание (на этот раз косое)

% Изменим формат \section и \subsection:
\titleformat{\section}
	{\vspace{1cm}\centering\LARGE\bfseries} % Стиль заголовка
	{}                                      % префикс
	{0pt}                                   % Расстояние между префиксом и заголовком
	{}                                      % Как отображается префикс
\titleformat{\subsection}                   	% Аналогично для \subsection
	{\Large\bfseries}
	{}
	{0pt}
	{}

% Поправленный вид lstlisting
\lstset { %
    backgroundcolor=\color{black!5}, % set backgroundcolor
    basicstyle=\footnotesize,% basic font setting
}

% Теоремы и утверждения. В комменте указываем номер лекции, в которой это используется.
\newtheorem*{hanoi_recurrent}{Свойство} % Лекция 1
\let\epsilent\varepsilon                % Лекция 8
\DeclareMathOperator{\rk}{rank}         % Лекция 20

% Информация об авторах
\author{Группа лектория ФКН ПМИ 2015-2016 \\
	Никита Попов \\
	Тамерлан Таболов \\
	Лёша Хачиянц}
\title{Лекции по предмету \\
	\textbf{Алгоритмы и структуры данных}}
\date{2016 год}


\begin{document}
\def\limref#1#2{{#1}\negmedspace\mid_{#2}}

\section{Лекция 24 от 14.03.2016}

\subsection*{Корневые подпространства}

Вспомним конец прошлой лекции.

Пусть $V$ --- векторное пространство над полем $\mathbb{F}$, $\phi \in L(V)$ --- линейный оператор.

Вектор $v \in V$ --- корневой для $\phi$, отвечающий собственному значению $\lambda \in \mathbb{F}$ тогда и только тогда, когда существует $m \geqslant 0 $ такое, что $(\phi - \lambda \id)^{m}(v) = 0.$ Высотой корневого вектора называется наименьшее такое $m$.

Корневым подпространством называется пространство из корневых векторов, соответствующих одному значению $\lambda$ и нулевого вектора. Другими словами, $V^{\lambda}(\phi) = \{v \in V \; \vline \; \exists m \geqslant 0 : (\phi - \lambda \id)^{m}(v) = 0\}$. Поскольку собственный вектор является корневым вектором высоты 1, то собственное подпространство включено в корневое подпространство с тем же значением: $V_{\lambda}(\phi) \subseteq V^{\lambda}(\phi).$

\begin{Suggestion}
	Корневое подпространство нетривиально тогда и только тогда, когда $\lambda$ является собственным значением. Другими словами, $V^{\lambda} \neq \{0\} 	
	\Leftrightarrow \chi_{\phi}(\lambda) = 0.$
\end{Suggestion}
\begin{proof}\
	\begin{itemize}
		\item[$\Leftarrow$]
		$\chi_{\phi}(\lambda) = 0 \Rightarrow V_{\lambda}(\phi) \neq \{0\} \Rightarrow V^{\lambda}(\phi) \neq \{0\}$, так как $V^{\lambda}(\phi) \supset V_{\lambda}(\phi)$.
		\item[$\Rightarrow$]
		Пусть $V^{\lambda}(\phi) \neq \{0\} \Rightarrow \exists v \neq 0 \in V^{\lambda}(\phi) \Rightarrow \exists m \geqslant 1: (\phi - \lambda \id)^{m}(v) = 0$. \\
		Рассмотрим $u = (\phi - \lambda \id)^{m - 1}(v) \neq 0,$ тогда:
		\begin{gather*}
		(\phi - \lambda \id)(u) = (\phi - \lambda \id)(\phi - \lambda \id)^{m - 1}(v) = (\phi - \lambda \id)^{m}(v) = 0.
		\end{gather*}
		То есть вектор $u$ --- это вектор, для которого $(\phi - \lambda \id)(u) = 0$, то есть собственный вектор. Следовательно $\lambda$ --- собственное значение.
	\end{itemize}
\end{proof}
\begin{Suggestion}
	Для любого собственного значения $\lambda \in \mathbb{F}$ подпространство $V^{\lambda}(\phi)$ инвариантно относительно $\phi$.
\end{Suggestion}
\begin{proof}
	Пусть $v$ --- корневой вектор высоты $m$. Докажем, что $\phi(v)$ --- также корневой вектор. 
	
	Заметим, что если $u = (\phi - \lambda \id)(v),$ то $u$ --- корневой вектор высоты $m - 1$, и, соответственно, лежит в корневом пространстве:
	$$
	u = (\phi - \lambda \id)(v) = \phi(v) - \lambda{v} \in V^{\lambda}(\phi).
	$$
	Мы получили, что 
	$\phi(v) \in \lambda{v} + V^{\lambda}(\phi).$ 
	Но $\lambda{v} \in V^{\lambda}(\phi)$, то есть $\lambda{v} + V^{\lambda}(\phi) = V^{\lambda}(\phi)$ и $\phi(v) \in V^{\lambda}(\phi)$. Что и означает, что пространство инвариантно относительно оператора $\phi$.
\end{proof}
Положим для краткости, что $\phi - \lambda \id = \phi_{\lambda}$.

\vspace{0.2cm}
Заметим, что ядра степеней линейного оператора <<вкладываются>> друг в друга --- те векторы, которые стали нулевыми при применении линейного оператора $\phi^k_\lambda$, при применении линейного оператора $\phi_\lambda$ ещё раз так и остаются нулевыми, а также <<добиваются>> (переводятся в нулевые) некоторые ранее ненулевые векторы. Итого, получаем следующее:
\[
V_\lambda(\phi) = \ker\phi_\lambda \subset \ker\phi^2_\lambda \subset \ldots \subset \ker\phi^m_\lambda 
\subset \ldots
\]
Причём существует такое $m$, что $\ker\phi^m_\lambda = \ker\phi^{m + 1}_\lambda$, так как $V$ --- конечномерно и размерность его не может уменьшаться бесконечно. Выберем наименьшее такое $m$.
\begin{Task}
	Доказать, что для любого $s \geqslant 0$ выполняется равенство $\ker\phi^m_\lambda = \ker\phi^{m + s}_\lambda.$
\end{Task}
Заметим также, что $V^{\lambda}(\phi) = \ker\phi^m_\lambda.$ Пусть $k_i = \dim\ker\phi^i_\lambda$.
Тогда:
 $$ \dim{V_{\lambda}(\phi)} = k_1 < k_2 < \ldots < k_m =  \dim{V^{\lambda}(\phi)}.
 $$
 
 Будем обозначать как $\limref{\phi}{V}$ ограничение линейного оператора на пространство $V$.
 
\begin{Suggestion}\
	\begin{enumerate}
		\item Характеристический многочлен линейного отображения $\phi \; \vline_{V^{\lambda}(\phi)}$ равен $(t - \lambda)^{k_m}$.
		\item Если $\mu \neq \lambda$, то линейный оператор $\phi - \mu \id$ невырожден на  $V^{\lambda}(\phi)$.
	\end{enumerate}
\end{Suggestion}

\begin{proof}
	Напомним, что $k_i = \dim \ker \phi^i_\l$, для $i = 1, \ldots, m$. Пусть также $k_0 = 0$.
	
	Выберем базис $\mathbb{e} = (e_1, \ldots, e_{k_m})$ в $V^{\lambda}(\phi)$ так, чтобы
	$(e_1, \ldots, e_{k_i})$ также был базисом в  $\ker\phi^{i}_\lambda$. Тогда:
	\begin{gather*}
	A(\limref{\phi_\l}{V^\l(\phi)}, \e) = 
	\begin{pmatrix}
	  0 & * & * & \ldots & * & * \\
	  0 & 0 & * & \ldots & * & * \\
	  0 & 0 & 0 & \ldots & * & * \\
	  \vdots &\vdots &\vdots & \ddots & \vdots & \vdots\\
	  0 & 0 & 0 & \ldots & 0 & * \\
	  0 & 0 & 0 &\ldots & 0 & 0
	\end{pmatrix}, \quad \text{где } a_{ij} \in \Mat_{(k_i - k_{i - 1}) \times (k_j - k_{j - 1})}
	\end{gather*}
	
	Но тогда:
	\begin{gather*}
	A(\limref{\phi}{V^\l(\phi)}, \e) = A(\limref{\phi_\l}{V^\l(\phi)}, \e) + \l E =
	\begin{pmatrix}
  		A_1 & * & * & \ldots & * & * \\
  	    0 & A_2 & * & \ldots & * & * \\
  	    0 & 0 & A_3 & \ldots & * & * \\
  		\vdots &\vdots &\vdots & \ddots & \vdots & \vdots\\
  		0 & 0 & 0 & \ldots & A_{m-1} & * \\
  		0 & 0 & 0 & \ldots & 0 & A_m
 	\end{pmatrix}, \quad \text{где } A_i = \lambda E_{k_i - k_{i - 1}} \quad (*)
	\end{gather*}
	А значит, характеристический многочлен линейного отображения $\limref{\phi}{V^\l(\phi)}$ равен $(t - \lambda)^{k_m}$.
	
	Теперь докажем невырожденность линейного оператора $(\phi - \mu\id)$ при $\mu \neq \l$.
	
	Рассмотрим матрицу ограничения этого оператора на корневое подпространство:
	$$
	A(\limref{(\phi - \mu \id)}{V^\l(\phi)}, \e) = A(\limref{\phi}{V^\l(\phi)}, \e) - \mu E.
	$$
	
	Она имеет вид $(*)$, где $A_i = (\lambda - \mu)E_{k_i}$. Следовательно,
	$$
	\det(\limref{(\phi - \mu \id)}{V^\l(\phi)}) = (\lambda - \mu)^{k_m} \neq 0.
	$$
	Что и означает, что линейный оператор невырожден.
\end{proof}

\begin{Suggestion}
	Если $\lambda$ -- собственное значение $\phi$, то $\dim{V^{\lambda}(\phi)}$ равен кратности 
	$\lambda$ как корня многочлена $\chi_\phi(t)$.
\end{Suggestion}

\begin{proof}
	Пусть $(e_1, \ldots, e_k)$ --- базис $V^{\lambda}(\phi),\ k = \dim{V^{\lambda}(\phi)}$. Дополним $(e_1, \ldots, e_k)$  до базиса $\mathbb{e} = (e_1, \ldots, e_n)$ всего пространства $V$. Тогда матрица линейного оператора имеет следующий вид:
	\begin{gather*}
	A_\phi = 
		\begin{pmatrix}
		\begin{array}{c|c}
		B & * \\ \hline
		0 & C
		\end{array}
		\end{pmatrix},\quad B \in M_k,\ C \in M_{n-k} \\
		\chi_\phi(t) = \det(tE - A) = \det(tE - B)\det(tE - C).
	\end{gather*}
	
	Заметим, что $\det(tE - B)$ --- это характеристический многочлен $\limref{\phi}{V^\l(\phi)}$, следовательно, 
	$$
	\chi_\phi(t) = (t - \lambda)^k\det(tE - C).
	$$
	Осталось показать, что $\lambda$ --- не корень $\det(tE - C)$.
	
	Пусть $W = \langle e_{k+1}, \ldots, e_n \rangle$. Тогда рассмотрим линейный оператор $\psi \in L(W)$, у которого матрица в базисе $(e_{k+1}, \ldots, e_n)$ есть $C$. Предположим, что $\det(\lambda E - C) = 0.$ Это значит, что $\lambda$~--- собственное 	значение для $\psi$ и существует вектор $w \in W,\ w \neq 0$ такой, что $\psi(w) = \lambda w$.
	
	Тогда:
	\begin{gather*}
	\phi(w) = \l w + u, \quad u \in V^\l(\phi) \\
	\phi(w) - \l w \in V^\l(\phi) \\
	(\phi - \l\id)(w) \in V^\l(\phi) \Rightarrow w \in V^\l(\phi)
	\end{gather*}
	Получили противоречие. Значит, $\l$ --- не корень $(tE - C)$.
\end{proof}

\begin{Consequence}
	$V^{\lambda}(\phi) = \ker\phi^s_\lambda$, где $s$ --- кратность $\lambda$ как корня многочлена $\phi_\lambda(t)$.
\end{Consequence}

\begin{Suggestion}
	Если $\lambda_1, \ldots, \lambda_k$, где $\lambda_i \neq \lambda_j$ при $i \neq j$ --- собственные значения $\phi$, то сумма $V^{\lambda_1}(\phi) + \ldots + V^{\lambda_k}(\phi)$ --- прямая.
\end{Suggestion}

\begin{proof}
	Докажем индукцией по $k$.
	
	База при $k = 1$ --- ясно.
	
	Теперь пусть утверждение доказано для всех значений, меньших $k$. Докажем для $k$.
	
	Выберем векторы $v_i \in V^{\lambda_i}(\phi)$ такие, что $v_1 + \ldots + v_k = 0$. Пусть $m$ --- высота вектора $v_k$. Тогда применим к нашей сумме оператор $\phi^m_{\lambda_k}$, получив следующее:
	\[
	\phi^m_{\lambda_k}(v_1) + \ldots + \phi^m_{\lambda_k}(v_{k-1}) + \phi^m_{\lambda_k}(v_k) = 0.
	\]
	С другой стороны, $ \phi^m_{\lambda_k}(v_k) = 0$, то есть:
	\[
	\phi^m_{\lambda_k}(v_1) + \ldots + \phi^m_{\lambda_k}(v_{k-1}) + \phi^m_{\lambda_k}(v_k) = \phi^m_{\lambda_k}(v_1) + \ldots + \phi^m_{\lambda_k}(v_{k-1}) = 0.
	\]
	Тогда по предположению индукции $\phi^m_{\lambda_k}(v_1) = \ldots = \phi^m_{\lambda_k}(v_{k-1}) = 0$. Hо
	$ \limref{\phi_\l}{V^\l(\phi)}$ не вырожден и обратим при $i \neq k$, следовательно $v_1 = \ldots = v_{k-1} = 0$. Но тогда и $v_k = 0$. 
	
	Следовательно, сумма прямая, что нам и требовалось.
\end{proof}

\begin{Theorem}
	Если характеристический многочлен $\chi_\phi(t)$ разлагается на линейные множители, причём $\chi_\phi(t) = (t - \lambda_1)^{k_1}\ldots(t - \lambda_s)^{k_s}$, то $V = \bigoplus_{i = 1}^s  \phi^{\lambda_i}(\phi)$.
\end{Theorem}

\begin{proof}
	Так как сумма $ \phi^{\lambda_i}(\phi) + \ldots +  \phi^{\lambda_i}(\phi)$ прямая и для любого $i$ выполняется, что $\dim(\phi^{\lambda_i}(\phi)) = k_i$, то: 
	$$
	\dim(\phi^{\lambda_1}(\phi) + \ldots +  \phi^{\lambda_s}(\phi)) = k_1 + \ldots + k_s = \dim{V}.
	$$
	Следовательно, $V = \bigoplus_{i = 1}^s  \phi^{\lambda_i}(\phi)$.
\end{proof}

\subsection{Жордановы клетки}
\begin{Def}
Пусть $\lambda \in \mathbb{F}$. \textbf{Жордановой клеткой} порядка $n$, отвечающей значению $\lambda$, называется матрица вида:
\begin{gather*}
J_\l^n = 
\begin{pmatrix}
\l & 1 & 0 & \ldots & 0 & 0 \\
0 & \l & 1 & \ldots & 0 & 0 \\
0 & 0 & \l & \ddots & 0 & 0 \\
\vdots & \vdots & \vdots & \ddots & \ddots & \vdots \\
0 & 0 & 0 & \ldots & \l & 1 \\
0 & 0 & 0 & \ldots & 0 & \l
\end{pmatrix} \in M_n(\mathbb{F}).
\end{gather*}
\end{Def}

\end{document}
